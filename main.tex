
\documentclass[a4paper,11pt]{article}
\newlength{\outerbordwidth}
\pagestyle{empty}
\raggedbottom
\raggedright
\usepackage[svgnames]{xcolor}
\usepackage{framed}
\usepackage{tocloft}
\usepackage[utf8]{inputenc}
\usepackage{CJK}
%\usepackage{url}
\usepackage{hyperref}
\usepackage{tabu}
\usepackage{eurosym}
\usepackage{geometry}
\geometry{top=0.4 cm, bottom=0.4 cm, left=0.4cm , right=0.4 cm}

%-----------------------------------------------------------
%Edit these values as you see fit

\setlength{\outerbordwidth}{2pt}  % Width of border outside of title bars
\definecolor{shadecolor}{gray}{0.7}  % Outer background color of title bars (0 = black, 1 = white)
\definecolor{shadecolorB}{gray}{0.8}  % Inner background color of title bars


%-----------------------------------------------------------
%Margin setup

%\setlength{\evensidemargin}{0.9in}
%\setlength{\headheight}{0in}
%\setlength{\headsep}{0in}
%\setlength{\oddsidemargin}{-0.9in}
%\setlength{\paperheight}{11in}
%\setlength{\paperwidth}{8.5in}
%\setlength{\tabcolsep}{0in}
%\setlength{\textheight}{9.5in}
%\setlength{\textwidth}{7in}
%\setlength{\topmargin}{-0.3in}
%\setlength{\topskip}{0in}
%\setlength{\voffset}{0.1in}


%-----------------------------------------------------------
%Custom commands
\newcommand{\resitem}[1]{\item #1 \vspace{-2pt}}
\newcommand{\resheading}[1]{\vspace{-12pt}
  \parbox{\textwidth}{\setlength{\FrameSep}{\outerbordwidth}
    \begin{shaded}
\setlength{\fboxsep}{0pt}\framebox[\textwidth][l]{\setlength{\fboxsep}{4pt}\fcolorbox{shadecolorB}{shadecolorB}{\textbf{\sffamily{\mbox{~}\makebox[7.7in][l]{\large #1} \vphantom{p\^{E}}}}}}
    \end{shaded}
  }\vspace{-13pt}
}
\newcommand{\ressubheading}[4]{
\begin{tabular*}{7.4in}{l@{\cftdotfill{\cftsecdotsep}\extracolsep{\fill}}r}
		\textbf{#1} & #2 \\
		\textit{#3} & \textit{#4} \\
\end{tabular*}\vspace{-6pt}}
%-----------------------------------------------------------


\begin{document}
\begin{center}
\begin{tabu*} to \textwidth {  X[l]  X[c]  X[r]  }
%{8in}{l@{\extracolsep{\fill}}r}
\textbf{\Large WHANNOU Brian} &  & bwhannou@outlook.fr\\
Qualified Actuary, FRM I &  &  {\href{https://www.linkedin.com/in/bwhannou/}{www.linkedin.com/in/bwhannou/}}\\
 &  & ~~~~+41 7 82 29 04 69 \\

\end{tabu*}
\\
\end{center}

%%%%%%%%%%%%%%%%%%%%%%%%%%%%%%
\resheading{Professional experiences}
%%%%%%%%%%%%%%%%%%%%%%%%%%%%%%
\begin{itemize}
\item \ressubheading{Associate Director: Risk modeling and Analytics Specialist}{September 2021 -}{UBS}{Lausanne, Suisse}
{\small
\begin{itemize}
	\item{Developed, monitored credit risk models for various corporate portfolios (Financial Institutions, SME, Ultra Wealthy Individual...)}
	\item{Ensured models compliance with various regulatory frameworks (FINMA, ECB and FED)}
	
\end{itemize}}
\item \ressubheading{Consultant: Risk Model Developer}{August 2020 -}{Optima Economics}{Cotonou, Bénin}
{\small
\begin{itemize}
	\item{Developed credit risk models and conducted ad-hoc statistical analyses.}
	
\end{itemize}}
\item \ressubheading{Senior associate: Data scientist (Financial Risk Management)}{September 2018 - August 2021}{KPMG}{France, Netherlands, Ivory Cost}
{
\begin{itemize}
	\item{\textbf{Credit Risk}: Expertise in Python and SAS programming to develop and deploy scoring models based on supervised learning algorithms in the IFRS9 framework ; Carried out an in-depth review of Credit risk (LGD, PD and CCF) models in a Top tier European Financial Institution under the Targeted Review of Internal Models (TRIM); Interacted with credit business units to provide advice on model development.}
	\item{\textbf{Assets Liabilities Management}: Improved the NII Stress-test exercise by reducing human interventions in the pipeline between data gathering and reporting; Developed a data-visualization tool; Developed and implemented a Model Risk Management methodology for ALM models.}
	\end{itemize}

\item \ressubheading{Quantitative analyst: Credit risk}{May 2016-August 2018}{Société Générale}{Paris, France}
{\small
\begin{itemize}
	\item{Implemented a machine learning algorithm based on clustering to forecast future loss on unresolved defaults and presented the results to a panel of credit risk experts and professionals from European Financial Institutions.}
\end{itemize}
}

\item \ressubheading{Quantitative analyst: Counterparty Credit Risk}{july 2015-September 2015}{Crédit Agricole CIB}{Paris, France}
{\small
\begin{itemize}
	\item{Developed a simulation methodology to diffuse correlated risk factors (interest rate, exchange rate, inflation, stocks) using R and C++.}
	
\end{itemize}}
}
\end{itemize}

%%%%%%%%%%%%%%%%%%%%%%%%%%%%%%
\resheading{Education}
%%%%%%%%%%%%%%%%%%%%%%%%%%%%%%
\begin{itemize}


\item \ressubheading{\href{http://www.ensae.fr}{ENSAE Paristech}}{2013 - 2016}{Actuarial Science}{Paris, France}{\small
\begin{itemize}
	\resitem{life/non-life insurance : Regression for frequency (Poisson law) and severity (gamma, Pareto law), principle of provisioning (Chain ladder, IBNR), Survival model (Kaplan Meier, Cox Model).}
	\resitem{Risk Management: Value-at-Risk and Expected Shortfall; Backtest and stresstest}
	\resitem{Clustering and classification methods: random forest, logistic regression, LightGBM, Extreme values theory.}
\end{itemize}}

\item \ressubheading{University Paris VII}{2015 - 2016}{Master's degree Mathematical Statistics and Probability}{Paris, France}{\small
\begin{itemize}
	\resitem{Stochastics Processus applied to finance(Peter Tankov, PhD): Black-Scholes model, Vasicek model, delta-normal and full revaluation approaches for computing VaR.}
	\resitem{Statistical learning (Stéphan Clemençon, PhD): Loss functions, Support Vector Machine.}
\end{itemize}}
\end{itemize}


%%%%%%%%%%%%%%%%%%%%%%%%%%%%%%
\resheading{Skills}
%%%%%%%%%%%%%%%%%%%%%%%%%%%%%%

\begin{itemize}
\item \textbf{Software}\vspace{-8pt}{\small
\begin{itemize}
\item{\textbf{R} (Coursera); \textbf{Python} (IBM);}
%	\item{C++,SQL, Octave, Excel, VBA, SAS, \LaTeX;}
	\item{On my Github page, you can see some of the projects I have been working on (\href{https://github.com/BWhannou}{\textbf{https://github.com/BWhannou}})}
\end{itemize}}
\item \textbf{Languages}
	\vspace{-8pt}
	\begin{center}\begin{tabular*}{7.5in}{l@{\extracolsep{\fill}}r}
		\multicolumn{2}{c}{English \cftdotfill{\cftdotsep} Full professional proficiency }\\
        \multicolumn{2}{c}{
        \begin{CJK}{UTF8}{gbsn}
中文 (普通话) \cftdotfill{\cftdotsep} 汉语水平考试二级
\end{CJK}  }\\
		\vphantom{E}
\end{tabular*}
\end{center}\vspace*{-20pt}
\end{itemize}

%%%%%%%%%%%%%%%%%%%%%%%%%%%%%%
\resheading{Hobbies}
%%%%%%%%%%%%%%%%%%%%%%%%%%%%%%

\begin{itemize}
\item Free athlete, half-marathons (Valencia, Paris(3$\times$), Madrid, Lisbon), Paris 20 kms (3$\times$), Paris marathon.
\item Travel; reading (personal development books) and listening to podcasts (Seth Godin, Malcolm Gladwell).
\end{itemize}

\end{document}
